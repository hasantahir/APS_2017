\documentclass[conference, 10pt]{IEEEtran}


%
\usepackage[T1]{fontenc} % optional
\usepackage[cmex10]{amsmath}
\usepackage{calc}
\usepackage{amsfonts} % to load math symbols
\usepackage{mdwmath}
\usepackage{commath}
\usepackage{physics} % For using the oridnary derivative nomenclature
\usepackage{mdwmath}
\usepackage{mdwtab}
\hyphenation{op-tical net-works semi-conduc-tor}


\usepackage{graphicx}
\usepackage{color}
\usepackage{placeins}
\usepackage{float}
\usepackage{hyperref}
% % The following is done to hide ugly color boxes around the links
\usepackage{xcolor}
\hypersetup{
colorlinks,
linkcolor={red!50!black},
citecolor={blue!50!black},
urlcolor={blue!80!black}
}


\usepackage{booktabs}
\usepackage{standalone}
\usepackage{filecontents}

\usepackage{tabularx,colortbl}
\usepackage{pgfplots}
\usepackage{tikz}
\usepackage[americanresistors,americaninductors]{circuitikz}
\usepackage{tikz-dimline} % For dimensional drawing
\tikzset{every picture/.append style={font=\normalsize}}
\usepackage{relsize}

\tikzset{fontscale/.style = {font=\relsize{#1}}
}
\usetikzlibrary{positioning}
\usetikzlibrary{arrows}
\usetikzlibrary{patterns}
\pgfplotsset{compat=newest}
%% the following commands are sometimes needed
\usetikzlibrary{plotmarks}
\usepackage{grffile}
\usepackage{mathtools,amssymb,lipsum}



%%%%%%%%%%
%%%%%%%%%% TIPS and TRICKS
%%%%%%%%%%
%
% ------------------------------- Useful Tricks Learnt
% Use ={}& to align subequations to the left

% Use = for single equations

% Use &= for split equations

% Use commath package to properly write differential operators and derivatives.

% Use \int\limits to nicely put integral limits

% For long equations, use align environment with \notag\\ as a linebreak.

% To hide section numbers, place an asterisk after the section, e.g., \section*{}

% Put comments % in between the lines in order to avoid forming a new paragraph.

% To enter special characters into Inkspace figures, use Ctrl+U and then enter       the unicode. e.g., for \times symbol, the unicode is U+0D7. So the key entry would be Ctrl+U U+0d7 and then press enter.

% Put \eqref instead or \ref to reference equations. This will automatically put parantheses around the eq. number. amsmath package required.
%
% ----------------- To compile with references use the following order in Shell"
% 1. pdflatex filename.tex
% 2. bibtex filename (no extension)
% 3. bibtex filename (no extension)
% 4. pdflatex filename.tex
% -----------------

% Personal definitions
% Operators
\renewcommand{\v}[1]{\mathbf{#1}} % vectors
\newcommand{\ti}[1]{\tilde{#1}} % spectral representation

% Symbols
\renewcommand{\O}{\omega}  % omega
\newcommand{\E}{\varepsilon}  % epsilon
\renewcommand{\u}{\mu}  % mu
\newcommand{\p}{\rho}  % rho
\newcommand{\x}{\times}  % times
\renewcommand{\inf}{\infty}  % infinity
\newcommand{\infint}{\int\limits_{-\inf}^\inf} % integral by R
\renewcommand{\del}{\nabla}  % nabla operator
\renewcommand{\^}{\hat}  % unit vector
% \newcommand*\diff{\mathop{}\!\mathrm{d}} % Define differential operator







\begin{document}

\title{An Integral Equation Scheme for Plasma based Thin Sheets}


% author names and affiliations
% use a multiple column layout for up to three different
% affiliations
\author{\IEEEauthorblockN{Hasan T. Abbas}
\IEEEauthorblockA{Department of Electrical and\\
Computer Engineering\\
Texas A\&M University\\
College Station, TX 77843-3128\\
Email: hasantahir@tamu.edu}
\and
\IEEEauthorblockN{Robert D. Nevels}
\IEEEauthorblockA{Department of Electrical and\\
Computer Engineering\\
Texas A\&M University\\
College Station, TX 77843-3128\\
Email: nevels@ece.tamu.edu}
}
%%%%%%%%%%%%%%%
%%%%%%%%%%%%%%%%
%%%%%%%%%%%%%%%%
%%%%%%%%%%%%%%%%
% make the title area
\maketitle


%
\begin{abstract}
  %\boldmath
  An integral equation formulation for a thin dielectric sheet is presented using the surface equivalence theorem. The advantageous properties of plasma waves, chief among them miniaturization of antennas and microwave devices are briefly discussed. Numerical results are presented to illustrate the scattering properties of the sheet with different material properties.
\end{abstract}

\IEEEpeerreviewmaketitle
%%%%%%%%%%%%%%%
%%%%%%%%%%%%%%%%
%%%%%%%%%%%%%%%%
%%%%%%%%%%%%%%%%
\section{Introduction}

The emergence of high-precision nanoscale fabrication techniques has recently led to an increased interest in two-dimensional (2D) materials, especially in the terahertz frequency regime and superconductive devices. One particularly intriguing example is the two-dimensional electron gas (2DEG) existing in the multilayer stack of semiconductor structures like high-electron mobility transistors (HEMTs), with remarkable electrical properties such as very high values of free-electron densities as compared to bulk semiconductors. These free electrons form an extremely thin and conductive channel in the stack. We observe the scattering properties of the 2DEG by modeling it as an infinitesimally thin sheet of plasma. An interaction between an external electromagnetic radiation and plasma results in 2D plasmons (surface waves). In this paper, we formulate the scattering response of the plasma sheet surrounded by free-space using the surface equivalence theorem.
%%%%%%%%%%%%%%%
%%%%%%%%%%%%%%%%
%%%%%%%%%%%%%%%%
%%%%%%%%%%%%%%%%
\section{Theory}

\subsection{Surface Plasmons}
%
The electrical properties of any material can be characterized by a frequency-dependent permittivity:
%
\begin{equation}
  \E(\O)=\E_r - j\frac{\sigma(\O)}{\omega}
  \label{eq:epsilon}
\end{equation}
%
where $\E_r$ is the permittivity of the material at dc frequency and $\sigma$ is the conductivity given by a Drude-type model \cite{burke2000high}:
%
\begin{equation}
  \sigma(\O) = \frac{N e^2 \tau}{m^{\ast}}\frac{1}{1 + j \O \tau}
  \label{eq:conductivity}
\end{equation}
%
The parameters $e$ and $m^*$ are the charge and effective mass of an electron respectively, $N$ is free-charge density, and $\tau$ is the scattering time of free charges in the 2DEG determined by the electron mobility, $\mu_e$:
\begin{equation}
  \tau  = \frac{m^{\ast} \mu_e}{e}.
  \label{eq:tau}
\end{equation}
%
The dispersion relation for 2D plasma waves can be written in terms of the plasma frequency $\O_p$ and wave-number $k$:
%
\begin{equation}
  \O_{p} =  \sqrt{\frac{2 \pi e^2 N} {m^{\ast}} k}.
  \label{eq:N_2d}
\end{equation}
%
\subsection{Surface Integral Equation}

Consider a flat plasma sheet of length $L$ and thickness $t$ excited by a $\mathrm{TM_z}$ polarized plane wave as illustrated in Fig. \ref{fig:plate}. The plasma is assumed nonmagnetic and the dielectric constant is determined from \eqref{eq:epsilon}-\eqref{eq:N_2d}. By applying the surface equivalence theorem \cite[p. 328-333]{balanis2012advanced}, the plasma sheet is replaced by an equivalent set of surface electric and magnetic currents. For the case of 2DEG plasma, the thickness is treated as the limiting case where $t \to 0$. The resulting electric field integral equation (EFIE) in a homogeneous free-space is written as:
%
\begin{equation}
  E_i = \frac{\O \u}{4} \int \limits_{0}^{L} J_z(x') \left[ H_0^{(2)}(k_1 \mathrm r) + H_0^{(2)}(k_2 \mathrm r)\right] \mathrm{d}x'
  \label{eq:plateE}
\end{equation}
%
where $\mu$ is the free-space permeability, $J_z$ is the yet-unknown surface electric current, $H_n^{(2)}(\cdot)$ is the n-th order Hankel function of the second kind and $k_i$ with $i = 1,2$ are the corresponding wave-numbers of the free-space and plasma respectively, and $\mathrm r = |x - x'|$.
%
\begin{figure}[h]
  \normalsize
  \centering
  \includestandalone[width=.5\textwidth]{figures/tm_plate}
  \caption{Thin Plasma sheet under $\mathrm{TM_z}$ polarized plane wave}
  \label{fig:plate}
\end{figure}
%
The corresponding magnetic field in terms of the magnetic current $M_x$ is expressed as:
\begin{align}
  H_i = \frac{\O \u}{8}\int \limits_{0}^{L}  M_x(x')\left[ \E_1 H_0^{(2)}(k_1 \mathrm r) + \E_1 H_2^{(2)}(k_2 \mathrm r) \right.\notag\\
  \left. + \E_2 H_0^{(2)}(k_2 \mathrm r) + \E_2 H_2^{(2)}(k_2 \mathrm r) \right]\mathrm{d}x'
  \label{eq:plateH}
\end{align}
The expressions for $\mathrm{TE_z}$ polarization are duals of \eqref{eq:plateE} and \eqref{eq:plateH}.
%%%%%%%%%%%%%%%
%%%%%%%%%%%%%%%%
%%%%%%%%%%%%%%%%
%%%%%%%%%%%%%%%%
\section{Numerical Results}
%
A method of moments (MoM) solution using pulse basis functions with point matching is implemented to compute the currents in \eqref{eq:plateE} and \eqref{eq:plateH} respectively. The far-field is calculated using the Total Field Scattered Field (TFSF) technique determined by:
\begin{equation}
  \sigma_{\phi} \simeq \int \limits_{0}^{L} \left[J_z(x')\eta_1 + M_x(x')\sin(\phi_i)\right] e^{j k_1 x' \cos(\phi_i)} \mathrm{d}x'
  \label{eq:far-field}
\end{equation}
where $\eta_1$ is the free-space intrinsic impedance and $\phi_i$ is the angle of incidence.
%
We explore the scattering properties of Gallium Arsenide ($\mathrm{GaAs}/\mathrm{AlGaAs}$) and Strontium Titanate ($\mathrm{LaAlO_3}/\mathrm{SrTiO_3}$) based 2DEG plasma sheets where pertinent material data has been taken from measurements in \cite{burke2000high} and \cite{herranz2012high} respectively, and the results are compared with a PEC plate of same length. Fig. \ref{fig:rcs} shows that the backscatter cross-sections from Galium Arsenide and Strontium Titanate based sheets are reduced, but not appreciably. Although the result is expected to be lower than the perfect conductor, a relatively small difference illustrates the potential of these materials in applications such as plasmonic antennas and waveguides. In particular, a 2DEG made from Strontium Titanate with its higher dielectric constant appears to be a better choice.
%
\begin{figure}[h]
  \begin{center}
    \noindent
    \includegraphics[width=.5\textwidth]{figures/farfieldTM_2.tex}
    \caption{Backscattered fields from different sheets of length $2\lambda$ under $\mathrm{TM_z}$ plane wave incidence}
    \label{fig:rcs}
  \end{center}
\end{figure}
%

 In order to verify the integral equation proposed, we next consider a sheet of length $2 \lambda$ and dielectric constant of $4$ excited by a $\mathrm{TE}_z$ polarized plane wave. Fig. \ref{fig:rcs} shows the backscattered field computed from \eqref{eq:far-field} compared with a resistive-sheet model \cite{senior1987sheet} using Impedance Boundary Conditions (IBC). The results agree well until incident angle of $\pi/4$ below which the diffraction effects from the edges, due to finite thickness of the resistive sheet becomes significant.
%
\begin{figure}[h]
  \begin{center}
    \noindent
    \includegraphics[width=.5\textwidth]{figures/farfield_2.tex}
    \caption{Comparison of Backscattered fields from a sheet of length $2\lambda$ with $\E_2 = 4$ under $\mathrm{TE_z}$ polarization with \cite{senior1987sheet}}
    \label{fig:rcs}
  \end{center}
\end{figure}
% \begin{figure}[h]
%   \centering
%   \includestandalone[width=.5\textwidth]{figures/farfield_1255}
%   \caption{Backscattered fields from different sheets of length $1.25\lambda$}
%   \label{fig:rcs}
% \end{figure}
% %
\section{Conclusion}
%
We present a new class of surface integral equations for infinitesimally thin dielectric sheets based on the surface equivalence theorem. The prospects of 2DEG based antennas and plasmonic devices are investigated by exploring the scattering properties of the structures. Results are shown for different polarizations and compared with the state-of-the art literature. Additionally, material characterization using measurable physical quantities is outlined.
%
\bibliographystyle{IEEEtran}
% argument is your BibTeX string definitions and bibliography database(s)
\bibliography{mybib}
\end{document}
